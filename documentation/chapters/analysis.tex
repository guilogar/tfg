En este capítulo se realiza un estudio desde los requerimientos necesarios y profundizando las necesidades de la misma para cumplir con los objetivos que se esperan.

\newpage

\section{Requisitos}
En este apartado se indican los requisitos para el sistema a implementar. Así conseguiremos identificar las necesidades del mismo. Esto también nos servira para corroborar el cumplimiento de los requisitos establecidos.

En los siguientes apartados, mostraremos los requisitos funcionales, los requisitos no funcionales, los requisitos de información y los distintos diagramas de casos de uso.

\section{Objetivos del sistema}
A continuación, describimos los objetivos del sistema:

\begin{table}[!htbp]
    \centering
    \resizebox{\textwidth}{!}{
        \begin{tabular}{p{4cm} | p{10cm}}
          \hline
            OBJ-0001 & Acceso al sistema \\
          \hline
            Descripción & Se deberá controlar el acceso al sistema, mediante un sistema
                          de login y el cual a parte de permitir el acceso a las pantallas
                          de gestión, deberá controlar a nivel de petición el login. \\
           \hline
        \end{tabular}
    }
    \caption{OBJ-0001}
\end{table}

\begin{table}[!htbp]
    \centering
    \resizebox{\textwidth}{!}{
        \begin{tabular}{p{4cm} | p{10cm}}
          \hline
            OBJ-0002 & Gestionar Terrenos \\
          \hline
            Descripción & Se deberá permitir leer, borrar, crear y actualizar los
                          terrenos disponibles \\
           \hline
        \end{tabular}
    }
    \caption{OBJ-0002}
\end{table}

\begin{table}[!htbp]
    \centering
    \resizebox{\textwidth}{!}{
        \begin{tabular}{p{4cm} | p{10cm}}
          \hline
            OBJ-0003 & Gestionar Cultivos \\
          \hline
            Descripción & Se deberá permitir leer, borrar, crear y actualizar los
                          cultivos en los terrenos disponibles \\
           \hline
        \end{tabular}
    }
    \caption{OBJ-0003}
\end{table}

\begin{table}[!htbp]
    \centering
    \resizebox{\textwidth}{!}{
        \begin{tabular}{p{4cm} | p{10cm}}
          \hline
            OBJ-0004 & Gestionar Fitosanitarios \\
          \hline
            Descripción & Se deberá permitir leer, borrar, crear y actualizar los fitosanitarios                     aplicados a los cultivos de los terrenos disponibles \\
           \hline
        \end{tabular}
    }
    \caption{OBJ-0004}
\end{table}

\begin{table}[!htbp]
    \centering
    \resizebox{\textwidth}{!}{
        \begin{tabular}{p{4cm} | p{10cm}}
          \hline
            OBJ-0005 & Gestionar Riegos \\
          \hline
            Descripción & Se deberá permitir leer, borrar, crear y actualizar los
                          riegos sobre terrenos disponibles \\
           \hline
        \end{tabular}
    }
    \caption{OBJ-0005}
\end{table}

\begin{table}[!htbp]
    \centering
    \resizebox{\textwidth}{!}{
        \begin{tabular}{p{4cm} | p{10cm}}
          \hline
            OBJ-0006 & Gestionar Suscripción a Eventos Complejos \\
          \hline
            Descripción & Se deberá permitir la suscripción al evento complejo que se quiera
                          y escoger la acción a hacer por defecto cuando el evento complejo
                          se realiza. \\
           \hline
        \end{tabular}
    }
    \caption{OBJ-0006}
\end{table}

\begin{table}[!htbp]
    \centering
    \resizebox{\textwidth}{!}{
        \begin{tabular}{p{4cm} | p{10cm}}
          \hline
            OBJ-0007 & Consultar las notificaciones \\
          \hline
            Descripción & Se deberá permitir consultar el histórico de las notificaciones
                          del usuario, respecto a los eventos lanzados. \\
           \hline
        \end{tabular}
    }
    \caption{OBJ-0007}
\end{table}

\begin{table}[!htbp]
    \centering
    \resizebox{\textwidth}{!}{
        \begin{tabular}{p{4cm} | p{10cm}}
          \hline
            OBJ-0008 & Modificar el idioma de la aplicación \\
          \hline
            Descripción & Se deberá permitir la elección del idioma en la aplicación.
                          En concreto, entre castellano o inglés. \\
           \hline
        \end{tabular}
    }
    \caption{OBJ-0008}
\end{table}

\begin{table}[!htbp]
    \centering
    \resizebox{\textwidth}{!}{
        \begin{tabular}{p{4cm} | p{10cm}}
          \hline
            OBJ-0009 & Modificar la acción por defecto en eventos \\
          \hline
            Descripción & Se deberá permitir la elección del tipo de acción por defecto
                          en los eventos complejos. En concreto, entre Manual o Automático. \\
           \hline
        \end{tabular}
    }
    \caption{OBJ-0009}
\end{table}

\begin{table}[!htbp]
    \centering
    \resizebox{\textwidth}{!}{
        \begin{tabular}{p{4cm} | p{10cm}}
          \hline
            OBJ-0010 & Modificar el look and feel de la aplicación \\
          \hline
            Descripción & Se deberá permitir la elección del aspecto de la aplicación,
                          en concreto la elección será entre modo claro o modo oscuro. \\
           \hline
        \end{tabular}
    }
    \caption{OBJ-0010}
\end{table}

\begin{table}[!htbp]
    \centering
    \resizebox{\textwidth}{!}{
        \begin{tabular}{p{4cm} | p{10cm}}
          \hline
            OBJ-0011 & Búsqueda por palabras coincidentes \\
          \hline
            Descripción & Se deberá permitir en cada sección buscar y filtrar por palabras clave. \\
           \hline
        \end{tabular}
    }
    \caption{OBJ-0011}
\end{table}

\newpage

\section{Catálogo de actores}
En este apartado, mostraremos todos los actores implicados en el uso de la aplicación:

\begin{table}[!htbp]
    \centering
    \resizebox{\textwidth}{!}{
        \begin{tabular}{p{4cm} | p{10cm}}
          \hline
            ACT-001 & Usuario habitual \\
          \hline
            Descripción & Este usuario es el usuario final, es decir, el cliente. Dicho usuario
                          puede usar toda la aplicación en su totalidad. Es decir, tiene todos
                          los privilegios en el front-office. \\
           \hline
        \end{tabular}
    }
    \caption{OBJ-0011}
\end{table}

\section{Requisitos funcionales}
A continuación, exponemos los requisitos funcionales de nuestro sistema:

\begin{table}[!htbp]
    \centering
    \resizebox{\textwidth}{!}{
        \begin{tabular}{p{4cm} | p{10cm}}
          \hline
            UC-001 & Inicio de Sesión \\
          \hline
            Descripción & El sistema deberá mostrar el comportamiento que se describe a continuación
                          cuando un usuario desee iniciar sesión para acceder al sistema \\
            Precondición & El usuario debe existir con la contraseña especificada y con permisos para                 poder hacer el login \\
            Secuencia Normal (S) & 1. El sistema muestra la pantalla para iniciar sesión \\
                                 & 2. El usuario introduce sus credenciales en el formulario \\
                                 & 3. El sistema verifica que existe el usuario y que la contraseña es      correcta y muestra la pantalla principal del sistema. \\
            Postcondición & El usuario tiene la sesión iniciada con un tiempo máximo de una hora. \\
            Excepciones & 1. El usuario no existe y por tanto, el sistema da un error. \\
                        & 2. La contraseña no coincide con la del usuario, y por tanto el sistema
                             da un error. \\
           \hline
        \end{tabular}
    }
    \caption{UC-001}
\end{table}

\begin{table}[!htbp]
    \centering
    \resizebox{\textwidth}{!}{
        \begin{tabular}{p{4cm} | p{10cm}}
          \hline
            UC-002 & Cerrar Sesión \\
          \hline
            Descripción & El sistema deberá mostrar el comportamiento que se describe a continuación
                          cuando un usuario desee cerrar sesión para salir del sistema \\
            Precondición & El usuario debe existir y debe estar logueado. \\
            Secuencia Normal (S) & 1. El sistema muestra la pantalla principal del sistema \\
                                 & 2. El usuario pulsa la opción llamada ``Cerrar sesión'' \\
                                 & 3. El sistema detecta el cierre de sesión y expulsa al usuario
                                      del sistema \\
            Postcondición & El usuario tiene la sesión cerrada y ya no puede acceder al sistema \\
            Excepciones & 1. El usuario no tiene conexión a internet, y por tanto, la petición de
                             cerrar sesión no funciona. \\
           \hline
        \end{tabular}
    }
    \caption{UC-002}
\end{table}

\begin{table}[!htbp]
    \centering
    \resizebox{\textwidth}{!}{
        \begin{tabular}{p{4cm} | p{10cm}}
          \hline
            UC-003 & Listar Terrenos \\
          \hline
            Descripción & El sistema deberá mostrar el comportamiento que se describe a continuación
                          cuando un usuario desee listar sus terrenos \\
            Precondición & El usuario debe existir y debe estar logueado. \\
            Secuencia Normal (S) & 1. El sistema muestra la pantalla principal del sistema \\
                                 & 2. El usuario pulsa la opción llamada ``Terrenos'' \\
                                 & 3. El sistema detecta la petición y muestra los terrenos
                                      asociados al usuario. \\
            Postcondición & El usuario tiene un listado en la pantalla de sus terrenos asociados \\
            Excepciones & 1. El usuario no tiene conexión a internet, y por tanto, la petición
                             no funciona. \\
                        & 2. El usuario no tiene terrenos asociados, y por tanto, aparece un
                             listado vacío. \\
           \hline
        \end{tabular}
    }
    \caption{UC-003}
\end{table}

\begin{table}[!htbp]
    \centering
    \resizebox{\textwidth}{!}{
        \begin{tabular}{p{4cm} | p{10cm}}
          \hline
            UC-004 & Buscar Terrenos \\
          \hline
            Descripción & El sistema deberá mostrar el comportamiento que se describe a continuación
                          cuando un usuario desee buscar sus terrenos \\
            Precondición & El usuario debe existir y debe estar logueado. \\
            Secuencia Normal (S) & 1. El sistema muestra la pantalla principal del sistema \\
                                 & 2. El usuario pulsa la opción llamada ``Terrenos'' \\
                                 & 3. El usuario pulsa en el icono con forma de lupa \\
                                 & 4. El sistema detecta la petición y muestra los terrenos
                                      asociados al usuario que coinciden con la búsqueda. \\
            Postcondición & El usuario tiene un listado en la pantalla de sus terrenos asociados \\
            Excepciones & 1. El usuario no tiene conexión a internet, y por tanto, la petición
                             no funciona. \\
                        & 2. El usuario no tiene terrenos asociados, y por tanto, aparece un
                             listado vacío. \\
                        & 3. El usuario no tiene terrenos asociados que coincidan con la búsqueda,
                             y por tanto, aparece un listado vacío. \\
           \hline
        \end{tabular}
    }
    \caption{UC-004}
\end{table}

\begin{table}[!htbp]
    \centering
    \resizebox{\textwidth}{!}{
        \begin{tabular}{p{4cm} | p{10cm}}
          \hline
            UC-005 & Añadir Terreno \\
          \hline
            Descripción & El sistema deberá mostrar el comportamiento que se describe a continuación
                          cuando un usuario desee añadir un terreno \\
            Precondición & El usuario debe existir y debe estar logueado. \\
            Secuencia Normal (S) & 1. El sistema muestra la pantalla principal del sistema \\
                                 & 2. El usuario pulsa la opción llamada ``Terrenos'' \\
                                 & 3. El usuario pulsa en el icono con el símbolo ``+'' \\
                                 & 4. El sistema detecta la petición y muestra la pantalla
                                      de añadir terreno \\
                                 & 5. El usuario completa todos los campos y pulsa el botón de
                                      ``Añadir Terreno'' \\
            Postcondición & El usuario tiene un listado de sus terrenos con el terreno que acaba
                            de añadir \\
            Excepciones & 1. El usuario no tiene conexión a internet, y por tanto, la petición
                            no funciona. \\
                        & 2. El usuario no ha introducido de forma correcta algunos de los campos
                             y por tanto, la petición falla. \\
           \hline
        \end{tabular}
    }
    \caption{UC-005}
\end{table}

\begin{table}[!htbp]
    \centering
    \resizebox{\textwidth}{!}{
        \begin{tabular}{p{4cm} | p{10cm}}
          \hline
            UC-006 & Modificar Terreno \\
          \hline
            Descripción & El sistema deberá mostrar el comportamiento que se describe a continuación
                          cuando un usuario desee modificar un terreno \\
            Precondición & El usuario debe existir y debe estar logueado. \\
            Secuencia Normal (S) & 1. El sistema muestra la pantalla principal del sistema \\
                                 & 2. El usuario pulsa la opción llamada ``Terrenos'' \\
                                 & 3. El usuario pulsa en el icono con el símbolo de edición de
                                      color azul \\
                                 & 4. El sistema detecta la petición y muestra la pantalla
                                      de editar terreno \\
                                 & 5. El usuario modifica los campos que quiera y pulsa el botón de
                                      ``Modificar Terreno'' \\
            Postcondición & El usuario tiene un listado de sus terrenos con el terreno que acaba
                            de editar \\
            Excepciones & 1. El usuario no tiene conexión a internet, y por tanto, la petición
                             no funciona. \\
                        & 2. El usuario no ha introducido de forma correcta algunos de los campos
                             y por tanto, la petición falla. \\
           \hline
        \end{tabular}
    }
    \caption{UC-006}
\end{table}

\begin{table}[!htbp]
    \centering
    \resizebox{\textwidth}{!}{
        \begin{tabular}{p{4cm} | p{10cm}}
          \hline
            UC-007 & Borrar Terreno \\
          \hline
            Descripción & El sistema deberá mostrar el comportamiento que se describe a continuación
                          cuando un usuario desee borrar un terreno \\
            Precondición & El usuario debe existir y debe estar logueado. \\
            Secuencia Normal (S) & 1. El sistema muestra la pantalla principal del sistema \\
                                 & 2. El usuario pulsa la opción llamada ``Terrenos'' \\
                                 & 3. El usuario pulsa en el icono con el símbolo de borrado de
                                      color rojo \\
                                 & 4. El sistema detecta la petición, borra el terreno y
                                      actualiza el listado de terrenos \\
            Postcondición & El usuario tiene un listado de sus terrenos sin el terreno que acaba
                            de borrar \\
            Excepciones & 1. El usuario no tiene conexión a internet, y por tanto, la petición
                             no funciona. \\
           \hline
        \end{tabular}
    }
    \caption{UC-007}
\end{table}

\begin{table}[!htbp]
    \centering
    \resizebox{\textwidth}{!}{
        \begin{tabular}{p{4cm} | p{10cm}}
          \hline
            UC-008 & Listar Cultivos \\
          \hline
            Descripción & El sistema deberá mostrar el comportamiento que se describe a continuación
                          cuando un usuario desee listar los cultivos de sus terrenos \\
            Precondición & El usuario debe existir y debe estar logueado. \\
            Secuencia Normal (S) & 1. El sistema muestra la pantalla principal del sistema \\
                                 & 2. El usuario pulsa la opción llamada ``Cultivos'' \\
                                 & 4. El sistema detecta la petición, y lista los cultivos asociados
                                      a cualquier de sus terrenos \\
            Postcondición & El usuario tiene un listado de los cultivos de cualquiera de sus terrenos \\
            Excepciones & 1. El usuario no tiene conexión a internet, y por tanto, la petición
                             no funciona. \\
                        & 2. El usuario no tiene ningún terreno asociado y por tanto, el listado
                             aparece vacío. \\
                        & 3. El usuario no tiene ningún cultivo asociado y por tanto, el listado
                             aparece vacío. \\
           \hline
        \end{tabular}
    }
    \caption{UC-008}
\end{table}

\begin{table}[!htbp]
    \centering
    \resizebox{\textwidth}{!}{
        \begin{tabular}{p{4cm} | p{10cm}}
          \hline
            UC-009 & Buscar Cultivos \\
          \hline
            Descripción & El sistema deberá mostrar el comportamiento que se describe a continuación
                          cuando un usuario desee buscar los cultivos de sus terrenos \\
            Precondición & El usuario debe existir y debe estar logueado. \\
            Secuencia Normal (S) & 1. El sistema muestra la pantalla principal del sistema \\
                                 & 2. El usuario pulsa la opción llamada ``Cultivos'' \\
                                 & 3. El usuario pulsa en el icono con forma de lupa \\
                                 & 4. El usuario escribe la búsqueda que desea \\
                                 & 5. El sistema detecta la petición, y lista los cultivos asociados
                                      a cualquier de sus terrenos que coincida con la búsqueda \\
            Postcondición & El usuario tiene un listado de los cultivos de cualquiera de sus terrenos \\
            Excepciones & 1. El usuario no tiene conexión a internet, y por tanto, la petición
                             no funciona. \\
                        & 2. El usuario no tiene ningún terreno asociado y por tanto, el listado
                             aparece vacío. \\
                        & 3. El usuario no tiene ningún cultivo asociado y por tanto, el listado
                             aparece vacío. \\
                        & 4. El usuario no tiene ninguún cultivo o terreno que coincida con la
                             búsqueda \\
           \hline
        \end{tabular}
    }
    \caption{UC-009}
\end{table}

\begin{table}[!htbp]
    \centering
    \resizebox{\textwidth}{!}{
        \begin{tabular}{p{4cm} | p{10cm}}
          \hline
            UC-010 & Añadir Cultivo \\
          \hline
            Descripción & El sistema deberá mostrar el comportamiento que se describe a continuación
                        cuando un usuario desee añadir un cultivo a un terreno \\
            Precondición & El usuario debe existir y debe estar logueado. \\
            Secuencia Normal (S) & 1. El sistema muestra la pantalla principal del sistema \\
                                 & 2. El usuario pulsa la opción llamada ``Cultivos'' \\
                                 & 3. El usuario pulsa en el icono con el símbolo ``+'' \\
                                 & 4. El usuario selecciona el cultivo y el terreno que desea añadir \\
                                 & 5. El sistema detecta la petición y añadir el cultivo al terreno
                                      deseado \\
            Postcondición & El usuario tiene un listado de los cultivos de cualquiera de sus terrenos
                            con el cultivo añadido \\
            Excepciones & 1. El usuario no tiene conexión a internet, y por tanto, la petición
                             no funciona. \\
                        & 2. El usuario no tiene ningún terreno asociado y por tanto, el listado
                             aparece vacío. \\
           \hline
        \end{tabular}
    }
    \caption{UC-010}
\end{table}

\begin{table}[!htbp]
    \centering
    \resizebox{\textwidth}{!}{
        \begin{tabular}{p{4cm} | p{10cm}}
          \hline
            UC-011 & Modificar Listado de Cultivos \\
          \hline
            Descripción & El sistema deberá mostrar el comportamiento que se describe a continuación
                          cuando un usuario desee modificar un cultivo a un terreno \\
            Precondición & El usuario debe existir y debe estar logueado. \\
            Secuencia Normal (S) & 1. El sistema muestra la pantalla principal del sistema \\
                                 & 2. El usuario pulsa la opción llamada ``Cultivos'' \\
                                 & 3. El usuario pulsa en el icono con el símbolo de editar
                                      de color azul \\
                                 & 4. El usuario selecciona el cultivo y el terreno que desea añadir \\
                                 & 5. El sistema detecta la petición y modifica el listado de
                                      cultivos al terreno deseado \\
            Postcondición & El usuario tiene un listado de los cultivos de cualquiera de sus terrenos
                            con el nuevo listado de cultivos \\
            Excepciones & 1. El usuario no tiene conexión a internet, y por tanto, la petición
                             no funciona. \\
                        & 2. El usuario no tiene ningún terreno asociado y por tanto, el listado
                             aparece vacío. \\
           \hline
        \end{tabular}
    }
    \caption{UC-011}
\end{table}

\begin{table}[!htbp]
    \centering
    \resizebox{\textwidth}{!}{
        \begin{tabular}{p{4cm} | p{10cm}}
          \hline
            UC-012 & Listar Fitosanitarios \\
          \hline
            Descripción & El sistema deberá mostrar el comportamiento que se describe a continuación
                        cuando un usuario desee listar los fitosanitarios de sus cultivos \\
            Precondición & El usuario debe existir y debe estar logueado. \\
            Secuencia Normal (S) & 1. El sistema muestra la pantalla principal del sistema \\
                                 & 2. El usuario pulsa la opción llamada ``Fitosanitarios'' \\
                                 & 3. El sistema detecta la petición y lista los fitosanitarios \\
            Postcondición & El usuario tiene un listado de los fitosanitarios de los cultivos de
                            sus terrenos \\
            Excepciones & 1. El usuario no tiene conexión a internet, y por tanto, la petición
                             no funciona. \\
                        & 2. El usuario no tiene ningún terreno asociado y por tanto, el listado
                             aparece vacío. \\
                        & 3. El usuario no tiene ningún cultivo asociado y por tanto, el listado
                             aparece vacío. \\
           \hline
        \end{tabular}
    }
    \caption{UC-012}
\end{table}

\begin{table}[!htbp]
    \centering
    \resizebox{\textwidth}{!}{
        \begin{tabular}{p{4cm} | p{10cm}}
          \hline
            UC-013 & Buscar Fitosanitarios \\
          \hline
            Descripción & El sistema deberá mostrar el comportamiento que se describe a continuación
                          cuando un usuario desee listar los fitosanitarios de sus cultivos y
                          que coincida con la búsqueda \\
            Precondición & El usuario debe existir y debe estar logueado. \\
            Secuencia Normal (S) & 1. El sistema muestra la pantalla principal del sistema \\
                                 & 2. El usuario pulsa la opción llamada ``Fitosanitarios'' \\
                                 & 3. El usuario pulsa en el icono con forma de lupa \\
                                 & 4. El usuario escribe la búsqueda que desea \\
                                 & 5. El sistema detecta la petición, y lista los fitosanitarios
                                      con los cultivos asociados a cualquier de sus terrenos que
                                      coincida con la búsqueda \\
            Postcondición & El usuario tiene un listado de los fitosanitarios de los cultivos de
                            sus terrenos \\
            Excepciones & 1. El usuario no tiene conexión a internet, y por tanto, la petición
                             no funciona. \\
                        & 2. El usuario no tiene ningún terreno asociado y por tanto, el listado
                             aparece vacío. \\
                        & 3. El usuario no tiene ningún cultivo asociado y por tanto, el listado
                             aparece vacío. \\
                        & 4. El usuario no tiene ningún fitosanitario, cultivo o terreno que
                             coincida con la búsqueda. \\
           \hline
        \end{tabular}
    }
    \caption{UC-013}
\end{table}

\begin{table}[!htbp]
    \centering
    \resizebox{\textwidth}{!}{
        \begin{tabular}{p{4cm} | p{10cm}}
          \hline
            UC-014 & Añadir Fitosanitario \\
          \hline
            Descripción & El sistema deberá mostrar el comportamiento que se describe a continuación
                        cuando un usuario desee añadir un fitosanitario a uno de sus cultivos \\
            Precondición & El usuario debe existir y debe estar logueado. \\
            Secuencia Normal (S) & 1. El sistema muestra la pantalla principal del sistema \\
                                 & 2. El usuario pulsa la opción llamada ``Fitosanitarios'' \\
                                 & 3. El usuario pulsa en el icono con el símbolo ``+'' \\
                                 & 4. El sistema detecta la petición, y añade el fitosanitario al
                                      cultivo seleccionado. \\
            Postcondición & El usuario tiene un listado de los fitosanitarios de los cultivos de
                            sus terrenos con el fitosanitario añadido \\
            Excepciones & 1. El usuario no tiene conexión a internet, y por tanto, la petición
                             no funciona. \\
                        & 2. El usuario no tiene ningún terreno asociado y por tanto, el listado
                             aparece vacío. \\
                        & 3. El usuario no tiene ningún cultivo asociado y por tanto, el listado
                             aparece vacío. \\
           \hline
        \end{tabular}
    }
    \caption{UC-014}
\end{table}

\begin{table}[!htbp]
    \centering
    \resizebox{\textwidth}{!}{
        \begin{tabular}{p{4cm} | p{10cm}}
          \hline
            UC-015 & Modificar Listado de Fitosanitarios \\
          \hline
            Descripción & El sistema deberá mostrar el comportamiento que se describe a
                          continuación cuando un usuario desee modificar el listado de
                          fitosanitarios de un cultivo \\
            Precondición & El usuario debe existir y debe estar logueado. \\
            Secuencia Normal (S) & 1. El sistema muestra la pantalla principal del sistema \\
                                 & 2. El usuario pulsa la opción llamada ``Fitosanitarios'' \\
                                 & 3. El usuario pulsa en el icono de editar de color azul \\
                                 & 4. El sistema detecta la petición, y muestra la pantalla de
                                      edición de fitosanitarios \\
            Postcondición & El usuario tiene un listado de los fitosanitarios de los cultivos de
                            sus terrenos con la lista de fitosanitarios modificada \\
            Excepciones & 1. El usuario no tiene conexión a internet, y por tanto, la petición
                             no funciona. \\
                        & 2. El usuario no tiene ningún terreno asociado y por tanto, el listado
                             aparece vacío. \\
                        & 3. El usuario no tiene ningún cultivo asociado y por tanto, el listado
                             aparece vacío. \\
                        & 4. El usuario no tiene ningún fitosanitario asociado y por tanto, el listado
                             aparece vacío. \\
           \hline
        \end{tabular}
    }
    \caption{UC-015}
\end{table}

\begin{table}[!htbp]
    \centering
    \resizebox{\textwidth}{!}{
        \begin{tabular}{p{4cm} | p{10cm}}
          \hline
            UC-016 & Listar Riegos \\
          \hline
            Descripción & El sistema deberá mostrar el comportamiento que se describe a
                          continuación cuando un usuario desee listar los riegos asociados
                          a sus terrenos \\
            Precondición & El usuario debe existir y debe estar logueado. \\
            Secuencia Normal (S) & 1. El sistema muestra la pantalla principal del sistema \\
                                 & 2. El usuario pulsa la opción llamada ``Riegos'' \\
                                 & 3. El sistema detecta la petición y muestra el listado
                                      de los riegos \\
            Postcondición & El usuario tiene un listado de los riegos de sus terrenos \\
            Excepciones & 1. El usuario no tiene conexión a internet, y por tanto, la petición
                             no funciona. \\
                        & 2. El usuario no tiene ningún terreno asociado y por tanto, el listado
                             aparece vacío. \\
                        & 3. El usuario no tiene ningún riego asociado y por tanto, el listado
                             aparece vacío. \\
           \hline
        \end{tabular}
    }
    \caption{UC-016}
\end{table}

\begin{table}[!htbp]
    \centering
    \resizebox{\textwidth}{!}{
        \begin{tabular}{p{4cm} | p{10cm}}
          \hline
            UC-017 & Buscar Riegos \\
          \hline
            Descripción & El sistema deberá mostrar el comportamiento que se describe a
                          continuación cuando un usuario desee listar los riegos asociados
                          a sus terrenos que coincida con la búsqueda realizada \\
            Precondición & El usuario debe existir y debe estar logueado. \\
            Secuencia Normal (S) & 1. El sistema muestra la pantalla principal del sistema \\
                                 & 2. El usuario pulsa la opción llamada ``Riegos'' \\
                                 & 3. El usuario pulsa en el icono con forma de lupa \\
                                 & 4. El usuario escribe la búsqueda que desee. \\
                                 & 5. El sistema detecta la petición y lista los riegos de sus
                                      terrenos que coincida con la búsqueda \\
            Postcondición & El usuario tiene un listado de los riegos de sus terrenos que
                            coincida con la búsqueda \\
            Excepciones & 1. El usuario no tiene conexión a internet, y por tanto, la petición
                             no funciona. \\
                        & 2. El usuario no tiene ningún terreno asociado y por tanto, el listado
                             aparece vacío. \\
                        & 3. El usuario no tiene ningún riego asociado y por tanto, el listado
                             aparece vacío. \\
                        & 4. El usuario no tiene ningún terreno o riego asociado que coincida
                             con la búsqueda. \\
           \hline
        \end{tabular}
    }
    \caption{UC-017}
\end{table}

\begin{table}[!htbp]
    \centering
    \resizebox{\textwidth}{!}{
        \begin{tabular}{p{4cm} | p{10cm}}
          \hline
            UC-018 & Añadir Riego \\
          \hline
            Descripción & El sistema deberá mostrar el comportamiento que se describe a
                          continuación cuando un usuario desee añadir un riego
                          asociado a un terreno \\
            Precondición & El usuario debe existir y debe estar logueado. \\
            Secuencia Normal (S) & 1. El sistema muestra la pantalla principal del sistema \\
                                 & 2. El usuario pulsa la opción llamada ``Riegos'' \\
                                 & 3. El usuario pulsa en el icono con el símbolo ``+'' \\
                                 & 4. El usuario rellena los campos que se exigen \\
                                 & 5. El sistema detecta la petición y añade el riego al terreno
                                      seleccionado \\
            Postcondición & El usuario tiene un listado de los riegos de sus terrenos con
                            el riego añadido \\
            Excepciones & 1. El usuario no tiene conexión a internet, y por tanto, la petición
                             no funciona. \\
                        & 2. El usuario no tiene ningún terreno asociado y por tanto, el listado
                             aparece vacío. \\
                        & 3. El usuario no ha rellenado bien los campos exigidos y por tanto,
                             la petición falla. \\
           \hline
        \end{tabular}
    }
    \caption{UC-018}
\end{table}

\begin{table}[!htbp]
    \centering
    \resizebox{\textwidth}{!}{
        \begin{tabular}{p{4cm} | p{10cm}}
          \hline
            UC-019 & Modificar Riego \\
          \hline
            Descripción & El sistema deberá mostrar el comportamiento que se describe a
                          continuación cuando un usuario desee un riego con sus propiedades \\
            Precondición & El usuario debe existir y debe estar logueado. \\
            Secuencia Normal (S) & 1. El sistema muestra la pantalla principal del sistema \\
                                 & 2. El usuario pulsa la opción llamada ``Riegos'' \\
                                 & 3. El usuario pulsa en el icono de editar con el color azul \\
                                 & 4. El usuario modifica los campos como desee. \\
                                 & 5. El sistema detecta la petición y actualiza el riego
                                      con sus respectivas propiedades \\
            Postcondición & El usuario tiene un listado de los riegos de sus terrenos con
                            el riego modificado \\
            Excepciones & 1. El usuario no tiene conexión a internet, y por tanto, la petición
                             no funciona. \\
                        & 2. El usuario no tiene ningún terreno asociado y por tanto, el listado
                             aparece vacío. \\
                        & 3. El usuario no ha rellenado correctamente los campos y por tanto,
                             la petición falla. \\
           \hline
        \end{tabular}
    }
    \caption{UC-019}
\end{table}

\begin{table}[!htbp]
    \centering
    \resizebox{\textwidth}{!}{
        \begin{tabular}{p{4cm} | p{10cm}}
          \hline
            UC-020 & Borrar Riego \\
          \hline
            Descripción & El sistema deberá mostrar el comportamiento que se describe a
                          continuación cuando un usuario desee borrar un riego \\
            Precondición & El usuario debe existir y debe estar logueado. \\
            Secuencia Normal (S) & 1. El sistema muestra la pantalla principal del sistema \\
                                 & 2. El usuario pulsa la opción llamada ``Riegos'' \\
                                 & 3. El usuario pulsa en el icono de borrar de color rojo \\
                                 & 4. El sistema detecta la petición y borrar el riego
                                      seleccionado \\
            Postcondición & El usuario tiene un listado de los riegos sin el riego que se
                            acaba de borrar \\
            Excepciones & 1. El usuario no tiene conexión a internet, y por tanto, la petición
                             no funciona. \\
                        & 2. El usuario no tiene ningún terreno asociado y por tanto, el listado
                             aparece vacío. \\
                        & 3. El usuario no tiene ningún riego asociado y por tanto, el listado
                             aparece vacío. \\
           \hline
        \end{tabular}
    }
    \caption{UC-020}
\end{table}

\begin{table}[!htbp]
    \centering
    \resizebox{\textwidth}{!}{
        \begin{tabular}{p{4cm} | p{10cm}}
          \hline
            UC-021 & Listar Eventos Complejos \\
          \hline
            Descripción & El sistema deberá mostrar el comportamiento que se describe a
                          continuación cuando un usuario desee listar los eventos complejos
                          a los que esta suscrito \\
            Precondición & El usuario debe existir y debe estar logueado. \\
            Secuencia Normal (S) & 1. El sistema muestra la pantalla principal del sistema \\
                                 & 2. El usuario pulsa la opción llamada ``Eventos'' \\
                                 & 3. El sistema detecta la petición y lista los eventos complejos \\
            Postcondición & El usuario tiene un listado de los eventos complejos a los que el
                            usuario esta suscrito \\
            Excepciones & 1. El usuario no tiene conexión a internet, y por tanto, la petición
                             no funciona. \\
                        & 2. El usuario no tiene ningún evento complejo asociado, y por tanto,
                             el listado aparece vacío. \\
           \hline
        \end{tabular}
    }
    \caption{UC-021}
\end{table}

\begin{table}[!htbp]
    \centering
    \resizebox{\textwidth}{!}{
        \begin{tabular}{p{4cm} | p{10cm}}
          \hline
            UC-022 & Buscar Eventos Complejos \\
          \hline
            Descripción & El sistema deberá mostrar el comportamiento que se describe a
                          continuación cuando un usuario desee listar los eventos complejos
                          a los que esta suscrito y que coincida con la búsqueda deseada \\
            Precondición & El usuario debe existir y debe estar logueado. \\
            Secuencia Normal (S) & 1. El sistema muestra la pantalla principal del sistema \\
                                 & 2. El usuario pulsa la opción llamada ``Eventos'' \\
                                 & 3. El usuario pulsa en el icono con forma de lupa \\
                                 & 4. El usuario escribe la búsqueda deseada \\
                                 & 5. El sistema detecta la petición y lista los eventos
                                      complejos que coincida con la búsqueda \\
            Postcondición & El usuario tiene un listado de los eventos complejos a los que el
                            usuario esta suscrito y que coincide con la búsqueda realizada \\
            Excepciones & 1. El usuario no tiene conexión a internet, y por tanto, la petición
                             no funciona. \\
                        & 2. El usuario no tiene ningún evento complejo asociado, y por tanto,
                             el listado aparece vacío. \\
                        & 3. El usuario no tiene ningún evento complejo asociado que coincida
                             con la búsqueda realizada, y por tanto, el listado aparece vacío. \\
           \hline
        \end{tabular}
    }
    \caption{UC-022}
\end{table}

\begin{table}[!htbp]
    \centering
    \resizebox{\textwidth}{!}{
        \begin{tabular}{p{4cm} | p{10cm}}
          \hline
            UC-023 & Suscribirse a Eventos Complejos \\
          \hline
            Descripción & El sistema deberá mostrar el comportamiento que se describe a
                        continuación cuando un usuario desee suscribirse a un evento complejo \\
            Precondición & El usuario debe existir y debe estar logueado. \\
            Secuencia Normal (S) & 1. El sistema muestra la pantalla principal del sistema \\
                                 & 2. El usuario pulsa la opción llamada ``Eventos'' \\
                                 & 3. El usuario pulsa en el icono con el símbolo ``+'' \\
                                 & 4. El usuario selecciona el evento complejo y la acción
                                      automatizada a realizar \\
                                 & 5. El sistema detecta la petición, y suscribe al usuario al
                                      evento complejo \\
            Postcondición & El usuario tiene un listado de los eventos complejos a los que el
                            usuario esta suscrito con el evento que acaba de añadir \\
            Excepciones & 1. El usuario no tiene conexión a internet, y por tanto, la petición
                             no funciona. \\
                        & 2. El usuario no tiene ningún evento complejo asociado, y por tanto,
                             el listado aparece vacío. \\
                        & 3. El usuario a seleccionado un evento al que ya esta suscrito, y
                             por tanto, la petición falla. \\
           \hline
        \end{tabular}
    }
    \caption{UC-023}
\end{table}

\begin{table}[!htbp]
    \centering
    \resizebox{\textwidth}{!}{
        \begin{tabular}{p{4cm} | p{10cm}}
          \hline
            UC-024 & Desuscribirse de Eventos Complejos \\
          \hline
            Descripción & El sistema deberá mostrar el comportamiento que se describe a
                          continuación cuando un usuario desee desuscribirse de algún evento
                          complejo \\
            Precondición & El usuario debe existir y debe estar logueado. \\
            Secuencia Normal (S) & 1. El sistema muestra la pantalla principal del sistema \\
                                 & 2. El usuario pulsa la opción llamada ``Eventos'' \\
                                 & 3. El usuario pulsa el botón de borrar de color rojo \\
                                 & 4. El sistema detecta la petición y desuscribe al usuario
                                      del evento complejo \\
            Postcondición & El usuario tiene un listado de los eventos complejos a los que el
                            usuario esta suscrito sin el evento del que se acaba de desuscribir \\
            Excepciones & 1. El usuario no tiene conexión a internet, y por tanto, la petición
                             no funciona. \\
                        & 2. El usuario no tiene ningún evento complejo asociado, y por tanto,
                             el listado aparece vacío. \\
           \hline
        \end{tabular}
    }
    \caption{UC-024}
\end{table}

\begin{table}[!htbp]
    \centering
    \resizebox{\textwidth}{!}{
        \begin{tabular}{p{4cm} | p{10cm}}
          \hline
            UC-025 & Listar Notificaciones \\
          \hline
            Descripción & El sistema deberá mostrar el comportamiento que se describe a
                          continuación cuando un usuario desee listar el historico
                          de las notificaciones \\
            Precondición & El usuario debe existir y debe estar logueado. \\
            Secuencia Normal (S) & 1. El sistema muestra la pantalla principal del sistema \\
                                 & 2. El usuario pulsa la opción llamada ``Notificaciones'' \\
                                 & 3. El sistema detecta la petición y lista las notificaciones
                                      del usuario \\
            Postcondición & El usuario tiene un listado de todo el historico de las notificaciones,
                            ordenadas de más cercana a más lejana \\
            Excepciones & 1. El usuario no tiene conexión a internet, y por tanto, la petición
                             no funciona. \\
                        & 2. El usuario no tiene ninguna notificación, por tanto, el listado
                             aparece vacío. \\
           \hline
        \end{tabular}
    }
    \caption{UC-025}
\end{table}

\begin{table}[!htbp]
    \centering
    \resizebox{\textwidth}{!}{
        \begin{tabular}{p{4cm} | p{10cm}}
          \hline
            UC-026 & Buscar Notificaciones \\
          \hline
            Descripción & El sistema deberá mostrar el comportamiento que se describe a
                        continuación cuando un usuario desee listar el historico
                        de las notificaciones que coincida con la búsqueda \\
            Precondición & El usuario debe existir y debe estar logueado. \\
            Secuencia Normal (S) & 1. El sistema muestra la pantalla principal del sistema \\
                                 & 2. El usuario pulsa la opción llamada ``Notificaciones'' \\
                                 & 3. El usuario pulsa en el botón con forma de lupa \\
                                 & 4. El usuario escribe la búsqueda deseada \\
                                 & 5. El sistema detecta la petición, y lista las notificaciones
                                      que coincida con la búsqueda \\
            Postcondición & El usuario tiene un listado de todo el historico de las notificaciones,
                            ordenadas de más cercana a más lejana que coincida con la búsqueda \\
            Excepciones & 1. El usuario no tiene conexión a internet, y por tanto, la petición
                             no funciona. \\
                        & 2. El usuario no tiene ninguna notificación, por tanto, el listado
                             aparece vacío. \\
                        & 3. El usuario no tiene ninguna notificación que coincida con la búsqueda,
                             por tanto, el listado aparece vacío. \\
           \hline
        \end{tabular}
    }
    \caption{UC-026}
\end{table}

\begin{table}[!htbp]
    \centering
    \resizebox{\textwidth}{!}{
        \begin{tabular}{p{4cm} | p{10cm}}
          \hline
            UC-027 & Ver Ajustes \\
          \hline
            Descripción & El sistema deberá mostrar el comportamiento que se describe a
                        continuación cuando un usuario desee ver los ajustes del sistema \\
            Precondición & El usuario debe existir y debe estar logueado. \\
            Secuencia Normal (S) & 1. El sistema muestra la pantalla principal del sistema \\
                                 & 2. El usuario pulsa la opción llamada ``Ajustes'' \\
                                 & 3. El sistema detecta la petición, y muestra los ajustes del
                                      usuario \\
            Postcondición & El usuario tiene un listado de sus ajutes del sistema \\
            Excepciones & 1. El usuario no tiene conexión a internet, y por tanto, la petición
                             no funciona. \\
           \hline
        \end{tabular}
    }
    \caption{UC-027}
\end{table}

\begin{table}[!htbp]
    \centering
    \resizebox{\textwidth}{!}{
        \begin{tabular}{p{4cm} | p{10cm}}
          \hline
            UC-028 & Modificar Ajustes \\
          \hline
            Descripción & El sistema deberá mostrar el comportamiento que se describe a
                          continuación cuando un usuario desee modificar sus ajustes
                          del sistema \\
            Precondición & El usuario debe existir y debe estar logueado. \\
            Secuencia Normal (S) & 1. El sistema muestra la pantalla principal del sistema \\
                                 & 2. El usuario pulsa la opción llamada ``Ajustes'' \\
                                 & 3. El usuario modifica el ajuste que desee y le da al botón
                                      de guardar \\
                                 & 4. El sistema detecta la petición, y guarda las opciones
                                      modificadas \\
            Postcondición & El usuario tiene un listado con los ajustes del sistema actualizados
                            y el sistema se actualiza en vivo. \\
            Excepciones & 1. El usuario no tiene conexión a internet, y por tanto, la petición
                            no funciona. \\
           \hline
        \end{tabular}
    }
    \caption{UC-028}
\end{table}

\blankpage

\section{Requisitos no funcionales}
Los requisitos no funcionales de este proyecto son sacados del estándar ISO 25010 \cite{iso25010}, que son los siguientes:

\begin{itemize}
    \item Escalabilidad: se debe poder ampliar la aplicacion tanto a nivel lógico como a nivel físico.
    \item Disponibilidad: se debe permitir el acceso al sistema en cualquier momento y lugar.
    \item Usabilidad: la utilización del sistema debe ser algo sencillo e intuitivo. Además, ofreciendo funciones de utilidad como busquedas, copiar datos, entre otras.
    \item Seguridad: el sistema debe ser robusto, resistiendo contra ataques informáticos, por ende tanto la creación de usuarios como el login del mismo se hace mediante tokens de sesión únicos, irrepetibles y temporales.
    \item Compatibilidad: el sistema debe ser capaz de compartir información con otros, con independencia de la plataforma virtual o física en la cual este instalada.
    \item Mantenibilidad: el código del programa debe ser legible, modulado y fácilmente mantenible. Es decir, seguir los principios del Clean Code \cite{cleancode}.
\end{itemize}