\section{Manual de Instalación}

En este capítulo hablaremos sobre la instalación de las tecnologías necesarias y con sus configuraciones para arrancar el proyecto de forma satisfactoria.

Además, las tecnologías seleccionadas suelen tener un gran grado de retrocompatibilidad, así que, aunque se podrán versiones en concreto, muy seguramente con versiones más nuevas el proyecto funcionará igual.

\newpage

\subsection{Git}
Para instalar Git, solo tenemos que seguir los pasos del siguiente enlace \textcolor{blue}{\href{https://git-scm.com/book/en/v2/Getting-Started-Installing-Git}{Git}}

\subsection{Docker}
Con este sistema de virtualización, nos permite instalar versiones antiguas de software, que muchas veces con las actualizaciones de los sistema operativos, sin imposibles de actualizar.

Sobre todo, lo que más importa con eso son las base de datos, ya que son los casos más evidentes y más comunes. Veasé con el caso de MySQL 5.8.

Así pues, siguiendo las instrucciones de \textcolor{blue}{\href{https://docs.docker.com/engine/install}{Docker}}, instalaremos Docker en nuestro sistema.

\subsection{PostgreSQL}
Para instanciar una base de datos de PostgreSQL, usaremos lo instalado en el paso anterior y simplmente levantaremos el fichero docker-compose.yml y ya tenemos la base de datos que necesitamos.

\subsection{NVM}
Otro de los sistemas que necesitamos es NodeJS. Pero en vez de instalarnos el binario tal cual, usaremos esta tecnología llamada Node Version Manager.

Para instalar dicha tecnología, seguiremos las siguientes instrucciones \textcolor{blue}{\href{https://github.com/nvm-sh/nvm}{NVM}}.

Por último, para instalar NodeJS junto con su versión de NPM, solo tendremos que ejecutar el comando \textbf{nvm install 14}. Esto instala la versión de NodeJS 14.x.

\subsection{Ionic}
A parte de NodeJS y NPM, necesitamos el SDK de Ionic.

Este SDK lo podemos instalar con NPM, ejecutando el siguiente comando \textbf{npm install -g @ionic/cli}

Una vez lo tengamos instalado, solo tendremos que realizar el comando \textbf{ionic -v} para verificar dicha instalación.

\subsection{Código fuente}
Para obtener el código fuente de la aplicación, solo tenemos que hacer un git clone del repositorio. Dicho repositorio, es el siguiente: \textcolor{blue}{\href{https://github.com/guilogar/tfg}{https://github.com/guilogar/tfg}}.

Por tanto, si queremos hacer el clone, ejecutariamos el comando \\
\textbf{git clone https://github.com/guilogar/tfg}.

\subsection{Backend}
Para terminar la instalación del Backend, solo tendriamos que crear los archivos de credenciales necesarios.

Para esto, solo hay que crear 2 ficheros. Uno con variables de entorno genericas para la ejecución local y otro para las credenciales de Firebase.

A continuación, expongo el formato de los mismos.

\begin{lstlisting}[language=Java,caption={Fichero .env},captionpos=b]
DATABASE_NAME=smartrural
DATABASE_USERNAME=smartrural
DATABASE_PASSWORD=smartrural
DATABASE_HOST=127.0.0.1
DATABASE_DIALECT=postgres
ENVIRONMENT=develop

COSMOS_ENDPOINT=...
COSMOS_KEY=...
COSMOS_DATABASE=...
COSMOS_CONTAINER_ID=...
\end{lstlisting}

\begin{lstlisting}[language=Java,caption={Fichero firebaseServiceAccount.json},captionpos=b]
{
  "type": "service_account",
  "project_id": "",
  "private_key_id": "",
  "private_key": "",
  "client_email": "",
  "client_id": "",
  "auth_uri": "",
  "token_uri": "",
  "auth_provider_x509_cert_url": "",
  "client_x509_cert_url": ""
}
\end{lstlisting}

Por último, para instanciar el Backend, tan solo son necesarios 4 comandos.

\begin{itemize}
    \item \textbf{docker-compose up -d}
    \item \textbf{cd backend}
    \item \textbf{npm install}
    \item \textbf{node .}
\end{itemize}

\subsection{Frontend}
Respecto al Frontend, los ficheros de variables de entorno están versionados. Por tanto, para iniciar el frontend solo necesita 3 comandos.

\begin{itemize}
    \item \textbf{cd frontend}
    \item \textbf{npm install}
    \item \textbf{ionic serve}
\end{itemize}

\subsection{Datos por defecto}
Para poder usar el proyecto en local, se necesitan datos por defecto. Para poder generarlos, ya existe una funcionalidad que se ha programado que lo hace de forma automática.

Para insertar estos datos de ejemplo, se puede usar el comando \textbf{npm run makeNewData}.

De estos datos de ejemplo, se inserta un usuario con las siguientes credenciales:

\begin{itemize}
    \item Username: \textbf{test}
    \item Password: \textbf{test}
\end{itemize}

\subsection{Emulator}
Respecto al simulador de eventos simples, también se necesita crear un archivo de variables de entorno. A continuación, exponemos el formato del mismo.

\begin{lstlisting}[language=Java,caption={Fichero emulator/.env},captionpos=b]
CONNECTION_STRING=...
TIME_INTERVAL=30000
\end{lstlisting}

Para ejecutar el simulador, solo hace falta 3 comandos.

\begin{itemize}
    \item \textbf{cd emulator}
    \item \textbf{npm install}
    \item \textbf{node .}
\end{itemize}