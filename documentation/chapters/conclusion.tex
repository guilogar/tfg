Respecto a este último capítulo, mostraremos todo lo que se ha aprendido con este proyecto y el trabajo futuro del mismo.

\newpage

\section{Ámbito}
Respecto al ámbito, las conclusiones obtenidas han sido muy beneficiosas. Hemos conseguido con éxito elaborar una aplicación multiplataforma que permite al usuario gestionar todos sus recursos rurales e integra a la perfección con todas las nuevas tecnologías que actualmente imperan en el mercado.

\section{Aprendizaje}
Con este proyecto hemos aprendido como es el funcionamiento de una de las Cloud Públicas más relevantes en la actualidad, es decir, Azure Cloud.

Además, respecto al Backend, se ha aprendido a crear desde cero un sistema de login fiable y seguro, mediante un token de sesión. A parte, de diversos conocimientos adquiridos a la hora de conectar desde local con los servicios de Azure Cloud y como operar con dichos servicios dentro de operaciones de cron-tab.

Por otra parte, en el Frontend, se ha aprendido una tecnología que ahora mismo es una de las más importantes a saber del sector, como es React. Además, se ha aprendido usando la nueva forma de componentes funcionales, es decir, es lugar de crear objetos de clases, usando funciones.

Además, se ha aprendido respecto a usar el CI/CD que ofrece GitHub con su GitHub Workflow Action y como sincronizarlo con Azure para optimizar al máximo los despliegues. Como añadido, se ha usado esto que ofrece GitHub para además hacer compilaciones automáticas de esta documentación para mayor facilidad a la hora de obtenerla de forma sencilla para su entregable.

También, se ha aprendido a usar todo el soporte que ofrece Azure Cloud para un Sistema de IoT, Internet of Things, el cual es bastante completo.

\section{Problemas encontrados}
Los problemas encontrados han sido diversos.

El más destacado ha sido el precio de los servicios usados. Azure, como es normal, ofrece un buen servicio pero no a costo cero. Entonces, el desembolso que se ha tenido que hacer para poder usar la Cloud ha sido destacable.

Otro de los problemas encontrados, ha sido la integración de Stream Analytics con Azure CosmosDB, ya que, si se quería insertar información desde el primero al segundo, hacia falta hacer configuraciones un tanto complejas de las cuales apenas había información.

Por último, uno de los mayores problemas encontrados ha sido integrar Firebase de Google Cloud Platform con el Frontend para poder acceder a las Push Notifications tanto desde la plataforma web como desde el móvil.

Hemos de destacar que todos estos problemas se resolviero con éxito, pero han tomado bastante tiempo y horas dedicadas a la investigación de la documentación de por medio.

\section{Trabajo Futuro}
Por último, el presente proyecto tiene bastantes características deseables, pero le faltan algunas que, para un proyecto empresarial, serían sumamente imprescindibles.

Estas características son:

\begin{itemize}
    \item Mayor accesibilidad. Esta característica es sumamente importante, sobre todo cuando se trata de llegar al mayor número de clientes potenciales posibles. Uno de los principales elementos a implementar sería un sistema de TalkBack o Lector de Pantalla \cite{talkback}.
    \item Apartado de ayuda. Esta característica, como su propia nombre indica, se basaría en implementar un sistema de ayuda, incluso con un pequeño chat para ayudar al usuario a resolver dudas sobre el uso de la aplicación.
    \item Generar en Google Play Store y en Apple Store una aplicación del proyecto. Esta característica se implementaría únicamente para facilitar la descarga y la instalación de la aplicación en los dispositivos móviles de los usuarios.
    \item Crear más test tanto en Backend como en Frontend. Esta característica se implementaría usando las tecnologías de test de cada parte del proyecto y su beneficio sería aumentar la calidad y facilitar la comprobación de que el software funciona como debería en cada parte del mismo.
    \item Crear un BackOffice para un administrador total. Esta característica se implementaría en un Frontend a parte del usuario habitual. El principal beneficio sería el de diferenciar un administrador normal del administrador total que controla los recursos directamente en la nube.
    \item Cambiar el dispositivo que envía los eventos simples de un simulador a un dispositivo real, es decir, sensor más Raspberry Pi.
    \item Mejorar el CI/CD añadiendo la construcción y publicación de las aplicaciones móviles. Esta características se implementaría usando GitHub Workflow y el beneficio sería inmediato, es decir, eliminar al programador la engorrosa tarea de construir la aplicación y subirla posteriormente a la Play Store o a la App Store.
\end{itemize}